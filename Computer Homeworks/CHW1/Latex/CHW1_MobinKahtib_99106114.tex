\documentclass{article}
%\documentclass[12pt]{article}
\renewcommand{\baselinestretch}{1.4}
\usepackage{amsthm,amssymb,amsmath,graphicx}
\usepackage{color}
\usepackage[top=2cm, bottom=2cm, left=2.5cm, right=3cm]{geometry}
\usepackage[pagebackref=false,colorlinks,linkcolor=blue,citecolor=magenta]{hyperref}
\usepackage{xepersian}
\usepackage[euler]{textgreek}
\usepackage{graphicx}
\settextfont[Scale=1]{B Nazanin}
\graphicspath{ {./images/} }
\begin{document}
%1.1
\includegraphics{question1.jpg}\\\\
\includegraphics{figure1.jpg}\\\\
%1.2
\includegraphics{question2.jpg}\\\\\\
$y(t) - \alpha\times y(t - T_0) - \beta\times y(t - 2\times T_0) = x(t)$\\\\
$\mathcal{F}y(t) - \alpha\times\mathcal{F}y(t - T_0) - \beta\times \mathcal{F}y(t - 2\times T_0)= \mathcal{F}x(t)$\\\\
$Y(f)(1 - e^(-j2\pi fT_0)-e^(-j4\pi T_0)) = X(f)$\\\\ 
$H_c(f) = \frac{1}{ (1 - \alpha e^(-j2\pi fT_0)-\beta e^(-j4\pi f T_0))}$ \\\\
%1.3
\includegraphics{question3.jpg}\\\\
\includegraphics{figure3.jpg}\\\\
%1.4
\includegraphics{question4.jpg}\\\\
\includegraphics{figure4.jpg}\\\\
%1.5
\includegraphics{question5.jpg}\\\\
\begin{latin}
Here, if we have an analysis on the transformation function, we will see that by writing the Taylor expansions, the output includes
It is infinite from the input with different shifts and gains.

%2.1
\includegraphics{question21.jpg}\\\\

We replace  $\beta$ instead of $E[\beta]$ :
$y(t) - \alpha\times y(t - T_0) - {E[\beta]}\times y(t - 2\times T_0) = x(t)$\\\\
${E[\beta]} =  \sum_{n=1}^{\infty}  \beta* p_i(\beta)  = b_0 + b_1 + b_2 + ... + b_n$
$H_c(f) = \frac{1}{ (1 - \alpha e^(-j2\pi fT_0)-{E[\beta]} e^(-j4\pi f T_0))}$ \\\\
$H_c(f) = \frac{1}{ (1 - \alpha e^{-j2\pi fT_0}-(b_0 + b_1 + b_2 + ... + b_n) e^{-j4\pi f T_0})}$ \\\\
$H_{eq}(f) = ke^{-j2\pi ft_0}  (1 - \alpha e^{-j2\pi fT_0}-(b_0 + b_1 + b_2 + ... + b_n) e^{-j4\pi f T_0})$\\\\
$H_{eq}(f) = ke^{-j2\pi ft_0}  (1 - \alpha e^{-j2\pi fT_0}-b_0 e^{-j4\pi f T_0} - b_1e^{-j4\pi f T_0} - b_2 e^{-j4\pi f T_0} - ... - b_n e^{-j4\pi f T_0})$\\\\
Now we can say that   $e^{-j2\pi f T_0}$ for $\alpha$ and $e^{-j4\pi f T_0}$ for ${E[\beta]}$ are our delays which are created for different coefficients in each stage\\\\
%2.2
\includegraphics{question22.jpg}\\\\
$\gamma = 0.3$\\\\
$y(t) =  x(t) + \gamma x(t - T_0)$\\\\
$Y(f) = (1 + \gamma e^{-j2\pi fT_0})X(f)$\\\\ 
$H_c(f) = (1 + \gamma e^{-j2\pi fT_0})$\\\\
$H_{eq}(f) = \frac{ke^{-j2\pi ft_0}} {(1 + \gamma e^{-j2\pi fT_0})}$\\\\
$H_{eq}(f)=ke^{-j2 \pi ft_0}(\sum_{i=0}^{n} (−0.3)^i e^{ −j2\pi fT_0})$\\\\
Now with multiply Y(f) on both sides and get inverse fourier transform:\\\\
$x(t)=k\sum_{i=0}^{n} (-0.3)^iy(t - ( t_0+ iT_0))$\\\\
And therefore The size of the coefficients are:
1 , 0.3 ,0.09, 0.027,0.0081 ,0.00243,0.000729,0.0002187,etc\\\\
\end{latin}
%1.3
\includegraphics{question23.jpg}\\\\
\includegraphics{figure23.jpg}\\\\
%1.4
\includegraphics{question24.jpg}\\\\
\includegraphics{figure24.jpg}\\\\
\end{document}